\documentclass[11pt]{article}
\usepackage{amssymb,amsmath,tikz-cd,xspace, natbib,verse}
\usepackage[T1]{fontenc}

\begin{document}
\input{tex.macros}

\title{Category theory notes}
\maketitle

They say you don't understand something until you can explain it, so here is my exercise
in explaining category theory. I make no claims to any authority on this subject and the
document is incomplete in proportion to my incomplete understanding.

\section{Why}
The first question I have had to explain, to myself and every single person I've spoken
to about these things: why study category theory at all? This is forward looking,
but here are my impressions halfway through a few textbooks:

\items{
\item  By {\em category}, we mean {\em context}, and we're looking for patterns across
contexts. When two mathematical pursuits do the same thing, can
we formalize exactly how they are or are not identical?
There is a promise that even theoretical work in real-world
contexts (i.e., my actual job) will benefit from better formalization of how
contexts relate.
\item  I'm hoping it'll be a shortcut to learning more math. If I get this, then
algebraic topology should be metaphorical cake.
\item There is a trend in programming languages toward languages whose theory is based on
a category of types.  Out-of-fashion languages
are written in terms of procedures: plug 4 into the square functions and we do some
things and return 16. Languages oriented toward mathematical functions encourage a
thinking that is far more still: the square of 4 is 16 and always has been and always
will be.
\item From what I've read, a lot of philosophers work on it---there's some presumption that
it's not just about symbol shunting, but about finding `natural' and `universal'
properties of those patterns across disparate things.  The relationship
between these categories/contexts has always been there and we only need a way to see them.
Here's Robert Penn Warren on the matter:
}

\begin{verse}
Here is the shadow of truth, for only the shadow is true.\\
And the line where the incoming swell from the sunset Pacific\\
First leans and staggers to break will tell all you need to know\\
About submarine geography, and your father's death rattle\\
Provides all biographical data required for the {\em Who's Who} of the dead.

\dots In the distance, in {\em plaza, piazza, place, platz,} and square,\\
Boot heels, like history being born, on cobbles bang.

Everything seems an echo of something else.

\dots I watched the sheep huddling.  Their eyes\\
Stared into nothingness.  In that mist-diffused light their eyes\\
Were stupid and round like the eyes of fat fish in muddy water,\\
Or of a scholar who has lost faith in his calling. \dots

You would think that nothing would ever again happen.

That may be a way to love God.
\end{verse}

OK, now that we've covered the {\em why} question, let's start from simple element-to-element
mappings and work up.

\section{Leveling up}\label{levelsec}

A function $f:A\to B$, aka a mapping, assigns a value to all of the elements in the set $A$ to
some element(s) in the set $B$. Note that if $A=\{\emptyset\}$ then you still have a function, though
it is vacuous.

We can draw this function via arrows, maybe

\begin{center}
\begin{tikzcd}[row sep=tiny]
    a_1 \ar[rd] & b_1\\
    a_2 \ar[r] & b_2\\
    a_3 \ar[r] & b_3
\end{tikzcd}
\end{center}

Here, the nodes in the graph are single elements in the sets, and the arrows describe
where each element goes.

The function notation $f:A\to B$ is itself a tiny graph. If there were also a function
$g:C\to B$, we could construct a diagram that looks like the above element-by-element graph, but with sets at the nodes instead of single elements:

\Comdiag{A \ar[r,"f"]\& B\\
C \ar[ru, "g"] 
}
So we've gone up a level: each arrow at this level could be cracked open to reveal
a bundle of arrows, its own element-by-element diagram like the one we had above.

A closed function $f:A\to A$ maps from a set to itself, so we'd draw it as a loop:

\Comdiag{A\ar[loop right]}

I think a lot of math in the early- to mid-1900s worked at this level, defining as much as
possible via sets and mappings between sets. But we can bundle further. In the last
diagram we had a single function. But consider a collection of mappings (aka homomorphisms)
from set $A$ to set $B$, $\Hom{set}(A, B)$.
The first diagram with three elements in $A$ and
three in $B$ could be rewired nine different ways, but it's our choice and we can include
one to nine functions as desired.  We would draw these
bundles as
\Comdiag{A \ar[r,"\Hom{set}"]\& B}

I think of this arrow as a bundle of quivering wires, each with its own power. If this were anime
they'd be tentacles. But there is indeed a lot of power there: with \Re\ representing the
real numbers, we can now summarize all real one-variable functions with the diagram 
\Comdiag{\Re\ar[loop right]}

We are approaching the level at which category theory speaks. But we can go one step further, and put
this into a box as well, and now it is a single node in our graph. Let the above loop of
all functions $f:\Re\to\Re$ be $\bf R$, and let's keep its company with
the set of all functions on the natural numbers, $g:\Nat\to\Nat$, notated {\bf N}.

\Comdiag{{\bf N}\ar[r, "F"] \& {\bf R}}

Here, the arrow is a {\em functor}, which is a bundle of two sets of arrows at once:
one to map elements of $\Nat$ to elements of $\Re$, and one to map the functions
$\Nat\to \Nat$ to functions $\Re\to \Re$. [There are additional restrictions; see
below.] Custom is to write these with capital letters, to contrast with typically
lower-case functions.

All these diagrams look alike, so for any diagram you come across in a textbook, it's
worth checking where we are in the pyramid:
\items{
    \item Node is a single element; arrow represents a mapping from one element to another
    \item Node is a set; arrow represents a set of mappings from one set to another, aka a function
    \item Node is a set; arrow represents a set of functions from one set to another
    \item Node is a set of objects with their own sets of functions; arrow represents a functor from one context to another
}

\section{Bundles}\label{bundlesec}
We're used to thinking in terms of sets comprising atomic elements, but mathematical
objects are typically a bundle of parts. Non-mathematicians think about the non-negative real numbers
$\Re^+$, but we have to start thinking about ($\Re^+$ paired with addition) and ($\Re^+$
paired with multiplication) as different things. Some bundles:

\items{
    \item {\em Magma}: A set $S$ and a function $f:S\to S$.
    \item {\em Preorder}: A set $S$ and a binary relation $\leq$. (See flash cards for
the formal definition)
    \item {\em Graph}: A collection $(V, A, src, tgt)$:
    \items{
        \item $V$ a set (vertices)
        \item $A$ a set (arrows)
        \item $src:A\to V$ a function giving arrow sources
        \item $tgt:A\to V$ a function giving arrow targets
    }
}

Authors who write in object-oriented programming languages will recognize this
use of {\em object} to mean a structure of elements, some of which could be verbs.

\subsection{Constraints}
These objects often have constraints attached. For example, what we name the pair $(S, f)$ depends on
how restrictive we want to be about the function:

\items{
    \item {\em Magma}: A set $S$ and a function $f:S\to S$.
    \item {\em Semigroup}: A magma where $f$ is associative: $f(a, f(b,c))=f(f(a, b),c)$.
    \item {\em Monoid}: A semigroup with an identity element for which $f(id, s)=s,$ for all $s\in S$.
    \item {\em Group}: A monoid where every element has an inverse: $\forall x\in S$,
            $\exists x_{inv}$ such that $f(x, x_{inv}) = id$.
}

As an example, if we allow zero in $\Re^+$, then $(\Re^+, +)$ is a monoid (but no
negative numbers, so no inverses) and $(\Re^+, \cdot)$ is a group.

Associativity and identity are going to be part of the definition of categories,
so there won't be a lot of discussion of things on the list before monoids.
Groups are reserved for contexts where the sense of reversibility or symmetry make sense.

\subsection{Category constraints}
A category is a bundle of elements as per the end of the chain of arrow
abstractions. Bundle together some things that will be nodes (herein, objects), and
the sets of mappings between some or all of them.

You get to pick which mappings you will put in the set of mappings, with a few simple conditions.
You don't have to include every mapping (herein, morphism) from node to node, which as above may cover all of one-variable high school algebra with a single arrow. But for every set we do have to include
the identity morphism $id_A:A\to A$ mapping every element of an object to itself.

A category also has a composition function, so given $F:A\to B$
and $G:B\to C$, we have to include $G\circ F:A\to C$. If the composition function
is the usual $g(f(a)) = c$, so this is not difficult, but in case you get
creative, the textbooks also require that $id_b\circ f(a)=f(a)$ and $g\circ id(b)
= g(b)$.\footnote{Every author makes a big deal of how $G\circ F$ looks backward,
but all you have to do is read $\circ$ as {\em of}, as in {\em $g$ of $f$}, which is
exactly how people read $g(f(x))$. Really not a big deal at all.}

Having these additional rules means we can't just throw out any old set of nodes and edges
and call it a category, but it's not hard to add these extra mappings as needed,
to the point that people usually don't even bother drawing the loops to mark the
identities.

Functors, as above, are a bundle of mappings of objects to objects plus mappings from
functions to functions.  We want them to represent a sensible transformation from one
context to another, so a functor $F$ has to have these constraints on how it maps
functions and objects:
\items{
    \item $F(f:A\to B) = F(f): F(A)\to F(B)$
    \item $F(id_A) = id_{F(A)}$
    \item $f\circ g = F(f)\circ F(g)$
}

\section{Starting to map}
The nodes in all the graphs above could be anything, including the bundles of elements from
Section \ref{bundlesec}.

Let ${\bf PrO}$ be the set of all preorders and ${\bf Graph}$ be the set of graphs. A
single preorder is a set of relations of the form $a \leq b$, and we could replace the $\leq$
symbol with an arrow, like $a\leftarrow b$, and we have another little inline graph. 
Doing the same for all of the elements in the preorder would generate a list of
$($nodes, arrows, sources, targets$)$ in ${\bf Graph}$.
That is, $f:{\bf PrO}\to {\bf Graph}$ makes as much sense as $f:\Re \to \Re$. Set up such
a mapping for every element of ${\bf PrO}$, and we have a category diagram with two
objects and one arrow representing all homomorphisms mapping all partial orders to graphs.

A monoid as defined above is already a category. Objects are the set of elements $M$
on which the monoid is defined, we require a function $f:M\to M$ which must be
associative, and we require an identity mapping $f(id, a)= a$ (in fact, two, because
$f(a, id)$ also equals $a$). By definition, all the boxes are checked.

But we can also express a monoid in a manner that doesn't fit the intuitive approach
of having a set of elements, some functions between them, and the usual function
composition. I'm not clear on the benefits, but is is definitely cooler. Here are the
components of another category built from a monoid $(S, id, f)$:
\items{
    \item Objects: following the name {\em monoid}, this category has exactly one object.
Call it \S.
    \item Morphisms: For every monoid element $M$, define a morphism $M:\S\to\S$. Include
an identity element $id: \S\to\S$. That is, the same mapping, $f(\S)=\S$,
appears several times in the same category, distinguished by different labels.
    \item Composition: $M_1\circ M_2 \equiv f(M_1, M_2)$.
}
A category requires a composition function that is associative and correctly handles the
identity morphism, and we get exactly that from the monoid requirements on $f$. More on
this below.

Categories are themselves a bundle like the ones from Section \ref{bundlesec}: a set
of objects paired with a set of morphisms between them. So they too could be a node in
one of the above diagrams, giving us the category of categories, ${\bf Cat}$.\footnote{Authors refer
to {\em the} category of categories, and I'm unclear whether an alternative cat-of-cats
exists or what it would look like, so we may be at the top here---there's no category of
categories of categories.}

Things are already getting self-referential: ${\bf PrO}\to {\bf Graph}$ is a
node-and-arrow graph representing an operation involving the set of node-and-arrow
graphs. For ${\bf Cat}$, we can show that ${\bf Cat}\to {\bf Graph}$, thus drawing a
graph for two categories to represent how categories and graphs relate.  Mathematicians
love this stuff.

\section{Where commuting gets us} It's already potentially interesting that we can
express mappings between different concepts like orderings and graphs, but things don't get
interesting until we have multiple paths to the same place.

For example,
\citet{spivak:category} describes multisets as sets of elements $E$ with a set $B$ of
symbol names, and a map $\pi:E\to B$. So, for example, we could replace all the words in a
book with numbers, producing a sequence $e_1, e_2, \dots$, then have a dictionary
of the words, and $\pi$ tells us that $e_{250}$ maps to {\em dog}. We
may have another set of elements and symbols in another context, maybe a corpus in
another language, so we'll need an element mapping $f_1:E\to E'$ and a symbol mapping
$f_2:B\to B'$. See the flash cards for formalization.

Here is a diagram including both
 the functions to map a multiset $(E, B, \pi)$ to another, and
the $\pi$ functions inside the multisets:
\Comdiag{E \ar[r, "f_1"] \ar[d, "\pi"'] \& E'  \ar[d, "\pi'"]\\
        B \ar[r, "f_2"'] \& B'
    }

The dictionary definition of {\em
commute} is {\em to exchange}; in math circles, {\em commutativity} typically refers to
reversing the order of things, like $x+y = y+x$. In this case, we can 
get from the original element set to the primed symbol set in two ways:
either apply the symbol map $\pi$ first to get to $B$ and then step from the unprimed to
primed corpus (calculating $f_2\circ\pi(e)$), or first step to the primed corpus via $f_1$ and then
map from $E$ to $B$ on that side (calculating $\pi\circ f_1(e)$).

When are both routes identical? This is often an interesting question in its own right.
For books and word lists in Portuguese and English, I'd guess there's a natural
way to map simple nouns: finding that an element in the $P$ corpus is {\em cão},
then translating that word to the Portuguese-to-English dictionary to find that it means {\em dog}
is likely equivalent to the commuted procedure of first finding the element in the
English corpus matching our element of interest, then looking up that that element in
the $E$ corpus maps to {\em dog}. This is going to be easy and natural

Doing the same with {\em saudade} is going to take some forcing. It seems to me that
these exceptions are interesting in their own right: we're going to generate a list of
categories that all behave similarly and relate nicely, but what about those things
that have to be thrown out to fit the categorical mold? The textbooks that focus on
that mold naturally exclude them, but by definition, standardization requires jettisoning 
the unique.

Indeed, one way to nail down a relationship is to just cull down what is in your sets. The set
of books, $B$, can be cataloged either via the Library of Congress classification system
(a set of call numbers $L$), or the Dewey decimal system (another set of call numbers
$D$). Let us pair together call numbers referring to the same book, in the cross-product
space $D\times L$. The full cross product matching every Dewey number with ever LoC
number makes no sense, and not all books have a classification in every system,
so out of the full cross of all pairs $(d, l)$, we will write down the relatively
small list of pairs that are about the same book, herein $D\times_B L$.  Finally,
we have commutativity:
\Comdiag{D\times_B L \ar[r] \ar[d] \& L  \ar[d]\\
        D \ar[r] \& B
    }
This is a {\em pullback}. There is information in the pairing, and maybe even
information in the list of books and call numbers that can't be found from $D\times_B L$
and are now lost in the stacks.

\subsection{Natural transformations}
If there is a mapping from one context to another where (apply internal function,
then jump) produces the same outcome as (jump, then apply internal function), we call
the jump across categories a {\em natural transformation}. We had a set-level example
above where (find English lookup, translate lookup to Portuguese) and (find Portuguese
match to element in English corpus, lookup in Portuguese) had the same outcome. We can do
the same at the category level.

Given two categories {\bf C} and {\bf D}, we can define ${\rm Fun}({\bf C}, {\bf D})$ as
the category of functors between the two objects, and the set of arrows in that category
will be the natural transformations. This motivates defining and identifying natural
transformations, so we can bring the functors mapping {\bf C} to {\bf D} into the
categorical club.

Start with two categories and two functors, $F$ and $G$, each independently mapping the first
category to the second.  Pick two elements from the first category, $a$ and $b$, and a mapping
$h:a\to b$.  We want mappings $\alpha_a$ and $\alpha_b$ in the second category so that

\Comdiag{F(a)  \ar[r, "\alpha_a"] \ar[d, "F(h)"] \& F(a)  \ar[d, "G(h)"]\\
        G(a) \ar[r, "\alpha_b"] \& G(b)
    }

The functors are a natural transformation if, for any two elements of the first category,
we have $\alpha$s that make this square commute. Note that all of $F(a), F(h), G(b),
\dots$ are objects or mappings in the second category, and we require the $\alpha$s to
be as well. If we have all this, then we have the sort of naturalness we seek: you can
start in the $F$-transformed version and walk from $F(a)$ to $F(b)$ and then walk over to
the $G$-transformed version and get to $G(b)$; or you can commute the order and
$\alpha$-walk over to the $G$-transformed version first and then from $G(a)$ to $G(b)$.

Here is a failed attempt.
Consider the category consisting of a single object $A$, one named $B$, and an arrow $A\to
B$. We would like to map this category to ordered natural numbers, where the objects are
$1,2,\dots$ and the relation is $\leq$ by its usual meaning. Here are two functors we can
try to naturally transform:
\begin{eqnarray*}
    F(A) = 2 &;& F(B) = 5\\
    G(A) = 3 &;& G(B) = 4
\end{eqnarray*} 
These are valid functors, because both $2 \leq 5$ and $3 \leq 4$ are arrows that exist in
the target category.

But to have a natural transformation from the $F$ functor to the $G$ functor, we need
to find those two $\alpha$s to complete the square. We can't just make them up:
they have to be part of the set of mappings at the $(\Nat, \leq)$ category that both
functors map to. There is an arrow $2\to 3$ in that category, so we can do the walk
of $F(A) \to G(A) \to G(B)$, or $2 \leq 3 \leq 4$.  But we don't have $5\leq 4$,
so the walk from $F(A)\to F(B) \to G(B)$ isn't possible using the arrows the target
category gave us.

Given that all our arrows go uphill, we will need transformations where $F(A)$ is downhill
from both intermediate points $G(A)$ and $F(B)$, and $G(B)$ is uphill from both
intermediate points. For example, given
\begin{eqnarray*}
    F(A) = 2 &;& F(B) = 3\\
    G(A) = 4 &;& G(B) = 5
\end{eqnarray*} 
there is a natural transformation from $F$ to $G$ (but not from $G$ to $F$).

If we wanted to start with a category with more than two elements, we would need to show
that this relation holds for {\em any} pair of elements with an arrow between them.

\subsection{Chaining}
Further, all these commutative diagrams chain together. If we had English ($E$), Portuguese ($E'$), and
Russian  ($E''$) corpora, we could draw:

\Comdiag{E \ar[r, "f_1"] \ar[d, "\pi"'] \& E'   \ar[r, "f_3"] \ar[d, "\pi'"] \& E'' \ar[d, "\pi''"]\\
        B \ar[r, "f_2"'] \& B' \ar[r, "f_4"'] \&B''
    }

It is self-evident that if the left square and the right square commute, than any flow
from upper left to lower right will be equivalent. This becomes especially interesting
when we have mappings across different contexts: if we can map from multisets to preorders
and preorders to graphs, we just got a multiset-to-graph generator for free.

Natural transformations also chain: if there exists an n.t. from functor $F$ to functor
$G$, and from functor $G$ to functor $H$, then we have a natural transformation from $F$
to $H$. If we have three categories and natural transformations $F$ and $G$ from first to
second, and $H$ and $I$ from second to third, then $H\circ F$ must have a natural
transformation to $G\circ I$.

\subsection{Universal properties}
Consider the cross product of two sets, $A\times B$. There is an obvious mapping from
$A\times B \to A$ and $A\times B \to B$---just drop the unneeded $B$ or $A$ part. Here
is the shape of this as a diagram, with $\pi_1$ and $\pi_2$ being the {\em drop one
part} functions:

\Comdiag{
        \& X\times Y \ar[dl,"\pi_1"'] \ar[dr,"\pi_2"] \\
    X \& \& Y \\
}

This is called a span and can also be turned into a little inline diagram as
$X \leftarrow (X\times Y) \rightarrow Y$.
Of course, there are other things we could put at the top of the fork, some other
set $A$ that maps both ways: $X \leftarrow A \rightarrow Y$.
But $X\times Y$ is unique in this capacity because every $A$ and its span can be expressed
via a morphism to $X\times Y$. Here is the commutative diagram:

 \Comdiag{
        \& X\times Y \ar[dl,"\pi_1"'] \ar[dr,"\pi_2"] \\
    X \& \& Y \\
    \& A \ar[ul, "\forall f"] \ar[ur, "\forall g"'] \ar[uu, dashrightarrow, "f\times g"]
    }

As noted in the diagram, the $A\to (X\times Y)$ morphism is trivial to generate as
the ordered pair $(f(a), g(a))$.  For any $f$ and $g$, you could go the short way to
map $A$ to $X$ or $Y$, but you are guaranteed that there is also a pair of mappings
$A\to (X\times Y) \to X$ and $A\to (X\times Y) \to Y$ that are the long commute to the
same point.  Put differently, you can have an arbitrary span composed of any object and
mappings iff you can write down a map $A\to (X\times Y)$. That any span can be expressed
by going through $X\times Y$ makes that object a sort of universal spanner. Returning
to the problem of combining Dewey and LoC numbers, you could come up with all sorts
of other clever combination schemes besides the ordered pair $($Dewey, LoC$)$, but
you are guaranteed that all of them must reduce to this simpler scheme. This sounds
obvious, but people put a lot of time into trying to be clever.

One can generate other universal properties for other common diagrams and operations.
As per the names, the intent is that the universal properties do say something fundamental
about the relationship diagram, and that natural transformations do describe the
natural way to move from one domain to another.

\nocite{awodey:category}
\nocite{riehl:category}
\nocite{lawvere:conceptual}
\bibliographystyle{plainnat}
\bibliography{cat_notes}

\end{document}
