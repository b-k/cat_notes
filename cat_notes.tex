\documentclass[11pt]{article}
\usepackage{amssymb,amsmath,tikz-cd,xspace, natbib,verse}
\usepackage[T1]{fontenc}

\begin{document}
\input{tex.macros}

\title{Category theory notes}
\maketitle

Please see the Readme file for motivation and context. Right now the document only touches
on and motivates what seem to be the key ideas pushed by the basic category theory
textbooks I've been reading.

\section{Intro}
First question: why study category theory at all? Remember I'm a novice and this is
forward looking, but here are my impressions halfway through a few textbooks:

\items{
\item  I'm hoping it'll be a shortcut to learning more math. Like, if I get this, then
algebraic topology should be metaphorical cake.
\item  By {\em category}, we mean {\em context}, and we're looking for patterns across
contexts: a lot of mathematical pursuits sort of do the same thing over and over. Can
we formalize that and say that indeed, whatever two things that seem different are
actually the same thing?
\item  The intros seem to motivate it via verbs and motion, as if you are touring from category
to category, but maybe it's the opposite. Procedural code writers talk in terms of a
sequence of steps to get somewhere; functional people write functions that simply are.
It's not `plug 4 into the square functions and the computer does something and returns 16'
but `the square of 4 is 16 and always has been and always will be.' The relationship
between these categories/contexts has always been there and we only need a way to see
them.
\item From what I've read, a lot of philosophers work on it---there's some presumption that
it's not just about symbol shunting, but about finding `natural' and `universal'
properties of those patterns across disparate things. Here's Robert Penn Warren on the
matter:
}

\begin{verse}[\versewidth]
Here is the shadow of truth, for only the shadow is true.
And the line where the incoming swell from the sunset Pacific
First leans and staggers to break will tell all you need to know
About submarine geography, and your father’s death rattle
Provides all biographical data required for the {\em Who's Who} of the dead.

[...] and,
In the distance, in {\em plaza, piazza, place, platz,} and square,
Boot heels, like history being born, on cobbles bang.

Everything seems an echo of something else.

And when, by the hair, the headsman held up the head
Of Mary of Scots, the lips kept on moving,
But without sound.  The lips,
They were trying to say something very important.

But I had forgotten to mention an upland
Of wind-tortured stone white in darkness, and tall, but when
No wind, mist gathers, and once on the Sarré at midnight,
I watched the sheep huddling.  Their eyes
Stared into nothingness.  In that mist-diffused light their eyes
Were stupid and round like the eyes of fat fish in muddy water,
Or of a scholar who has lost faith in his calling.

Their jaws did not move.  Shreds
Of dry grass, gray in the gray mist-light, hung
From the side of a jaw, unmoving.

You would think that nothing would ever again happen.

That may be a way to love God.
\end{verse}

\section{Bundles}\label{bundlesec}
We're used to thinking in terms of sets comprising atomic elements, but mathematical
objects are typically a bundle of parts. Non-mathematicians think about the non-negative real numbers
$\Re^+$, but we have to start thinking about $\Re^+$ paired with addition and $\Re^+$
paired with multiplication as different things. Some bundles:

\items{
    \item {\em Magma}: A set $S$ and a function $f:S\to S$.
    \item {\em Preorder}: A set $S$ and a binary relation $\leq$. (See flash cards for
the formal definition)
    \item {\em Graph}: A collection $(V, A, src, tgt)$:
    \items{
        \item $V$ a set (vertices)
        \item $A$ a set (arrows)
        \item $src:A\to V$ a function giving arrow heads
        \item $tgt:A\to V$ a function giving arrow tails
    }
}

Authors who write in object-oriented programming languages will recognize this
use of {\em object} to mean a structure of elements, some of which could be verbs.

\subsection{Those object-function pairs}
These objects often have constraints attached, and what we name the pair $(S, f)$ depends on
how restrictive we want to be about the function:

\items{
    \item {\em Magma}: A set $S$ and a function $f:S\to S$.
    \item {\em Semigroup}: A magma where $f$ is associative: $f(a, f(b,c))=f(f(a, b),c)$.
    \item {\em Monoid}: A semigroup with an identity element for which $f(id)=id$.
    \item {\em Group}: A monoid where every element has an inverse: $\forall x\in S$, $\exists x_{inv}$ $\ni f(x, x_{inv}) = id$.
}

Associativity and identity are going to be part of the definition of categories,
so there won't be a lot of discussion of things on the list before monoids.
Groups are reserved for contexts where the sense of reversibility or symmetry make sense.

BTW, if we allow zero in $\Re^+$, then $(\Re^+, +)$ is a monoid (no negative numbers
$\rightarrow$ no inverses) and $(\Re^+, \cdot)$ is a group.

\section{Leveling up}\label{levelsec}

A function $f:A\to B$, aka a mapping, assigns a value to all of the elements in the set $A$ to
some element(s) in the set $B$. Note that if $A=\{\emptyset\}$ then you still have a function, though
it is vacuous.

We can draw this function via arrows, maybe

\begin{center}
\begin{tikzcd}[row sep=tiny]
    a_1 \ar[rd] & b_1\\
    a_2 \ar[r] & b_2\\
    a_3 \ar[r] & b_3
\end{tikzcd}
\end{center}

Here, the nodes in the graph are single elements in the sets, and the arrows describe
where each element goes.

The function notation $f:A\to B$ is itself a tiny graph. If there were also a function
$g:C\to B$, we could construct a diagram that looks like the above element-by-element graph:

\Comdiag{A \ar[r,"f"]\& B\\
C \ar[ru, "g"] 
}
So we've gone up a level: each arrow at this level could be cracked open to reveal
a bundle of arrows, its own element-by-element diagram like the one we had above. A
closed function $f:A\to A$ maps to its own set, so we'd draw it as a loop:

\Comdiag{A\ar[loop right]}

But we can bundle further. In the last diagram we had a single function. But
consider the collections of all
mappings of the form $f:A\to B$. The first diagram with three elements in $A$ and
three in $B$ could be rewired nine different ways. The full set of these mappings (aka
homomorphisms) is $\Hom{set}(A, B)$. But our collection doesn't have to be the whole set;
it's our choice. Sets use capitals and their elements lower case $(s\in S)$, so
given lower-case functions $f$, let a collection of those be $F$; we would draw these
bundles as

\Comdiag{A \ar[r,"F"]\& B}

I think of $F$ as a bundle of quivering wires, each with its own power. If this were anime
they'd be tentacles. But there is indeed a lot of power there: with \Re\ representing the
real numbers, we can now summarize all real one-variable functions with the diagram 
\Comdiag{\Re\ar[loop right]}

This is the level at which category theory speaks.

All these diagrams look alike, so for any random diagram out of a textbook, it's
worth checking where we are in the pyramid:
\items{
    \item Node is a single element; arrow represents a mapping from one element to another
    \item Node is a set; arrow represents a set of mappings from one set to another, aka a function
    \item Node is a set; arrow represents a set of functions from one set to another
}

\subsection{Category characteristics}
See the flash cards for the formalization, but less formally,
a category is a bundle of elements as per the end of that chain: some sets
(referred to as the objects in the category), and the sets of mappings between some
or all of them.  It also has a composition function, so given $F:A\to B$
and $G:B\to C$, then $G\circ F:A\to C$ is a thing. I haven't yet seen this composition
function be anything but the obvious $g(f(a)) = c$, so this is not difficult, but remember
that we select which morphisms (aka maps or functions) we want to include. If we have $F$
and $G$ as above, then including $G\circ F$ is not a choice.

A proper category also has some
constraining characteristics. First, we don't have to include the full set of mappings from
$A\to A$, which as above may cover vast territory, but for every set we do have to include
the identity morphism mapping every element of an object to itself. Second, the composition
is associative and correctly handles the identity mappings.

Having these additional rules means we can't just throw out any old set of nodes and edges
and call it a category, but we're usually going with the composition function of {\em apply $f$, then
apply $g$ to whatever you got out of $f$}, which works fine, and appending identity
morphisms is so easy to do that people usually don't even bother drawing the loops in,
so whatever diagram of sets and sets of functions you write down probably fits the rules.
Just as a group behaves differently from a monoid because of the additional structure,
the additional formality will allow us to say more below. But there does seem to be a
lot in common across levels, and many things that are true on the sets-and-functions
level will also work out on the cats-and-functors level.

\subsection{Starting to map}
The nodes in all the graphs above could be anything, including the bundles of elements from
Section \ref{bundlesec}.

Let ${\bf PrO}$ be the set of all preorders and ${\bf Graph}$ be the set of graphs. A
single preorder is a set of relations of the form $a \leq b$, and we could replace the $\leq$
symbol with an arrow, and $a\leftarrow b$ is already another little inline graph. 
Doing the same for all of the elements in the preorder would generate a list of
$($nodes, arrows, starting points, end points$)$ in ${\bf Graph}$.
That is, $f:{\bf PrO}\to {\bf Graph}$ makes as much sense as $f:\Re \to \Re$. Set up such
a mapping for every element of ${\bf PrO}$, and we have a category diagram with two
objects and homomorphisms between them.

Categories are themselves a bundle like the ones from Section \ref{bundlesec}: a set
of objects paired with a set of morphisms between them. So they too could be a node in
one of the above diagrams, giving us the category of categories.\footnote{Authors refer
to {\em the} category of categories, and I'm unclear whether an alternative cat-of-cats
exists or what it would look like, so we may be at the top here---there's no category of
categories of categories.}

Things are already getting self-referential: ${\bf PrO}\to {\bf Graph}$ is a node-and-arrow
graph representing an operation involving the set of node-and-arrow graphs. For the 
category of categories, ${\bf Cat}$, we can show that ${\bf Cat}\to {\bf Graph}$, thus
drawing a graph for the category \{${\bf Cat}, {\bf Graph}$\} to represent how categories and graphs relate.
Mathematicians love this stuff.

\section{Where commuting gets us} It's already potentially interesting that we can
express mappings between different concepts like orderings and graphs, but things don't get
interesting until we have multiple paths to the same place.


For example,
\citet{spivak:category} describes multisets as sets of elements $E$ with a set $B$ of
symbol names, and a map $\pi:E\to B$. So, for example, we could replace all the words in a
book with numbers, producing a sequence $e_1, e_2, \dots$, then have a dictionary
of the words, and $\pi$ tells us that $e_{250}$ is {\em dog}. We
may have another set of elements and symbols in another context, maybe a corpus in
another language, so we'll need an element mapping $f_1:E\to E'$ and a symbol mapping
$f_2:B\to B'$.

Here is a diagram
from the flash card ({\em qv}) on mapping a multiset $(E, B, \pi)$ to another:
\Comdiag{E \ar[r, "f_1"] \ar[d, "\pi"'] \& E'  \ar[d, "\pi'"]\\
        B \ar[r, "f_2"'] \& B'
    }

The dictionary definition of {\em
commute} is {\em to exchange}; in math circles, {\em commutativity} typically refers to
reversing the order of things, like $x+y = y+x$. In this case, we can 
get from the original element set to the primed symbol set in two ways:
either apply the symbol map $\pi$ first to get to $B$ and then step from the unprimed to
primed corpus (calculating $f_2\circ\pi(e)$), or first step to the primed corpus via $f_1$ and then
map from $E$ to $B$ on that side (calculating $\pi\circ f_1(e)$).

When are both routes identical? This is often an interesting question in its own right.
For books and word lists in Portuguese and English, I'd guess there's a natural
way to map simple nouns: Finding that an element in the $P$ corpus is {\em cão},
then translating that word to the Portuguese-to-English dictionary to find that it means {\em dog}
is likely equivalent to first finding the element in the English corpus matching our
element of interest, then looking up that that element in the $E$ corpus maps to {\em
dog}. This is going to be easy and natural, while doing the same with {\em saudade}
is going to take some forcing.

This example was at the set level, but one could do the same at the category level.
If there is a mapping from one world to another where (apply internal function, then jump)
produces the same outcome as (jump, then apply internal function), we call the jump across
categories a {\em natural transformation}.

Further, these diagrams chain together:

\Comdiag{E \ar[r, "f_1"] \ar[d, "\pi"'] \& E'   \ar[r, "f_3"] \ar[d, "\pi'"] \& E'' \ar[d, "\pi''"]\\
        B \ar[r, "f_2"'] \& B' \ar[r, "f_4"'] \&B''
    }

It is self-evident that if the left square and the right square commute, than any flow
from upper left to lower right will be equivalent. This becomes especially interesting
when we have mappings across different contexts: if we can map from multisets to preorders
and preorders to graphs, we just got a multiset-to-graph generator for free.

\subsection{Universal properties}
Consider the cross product of two sets, $A\times B$. There is an obvious mapping from
$A\times B \to A$---just drop the $B$ part---and $A\times B \to B$. Here is the shape of this as a
diagram, with $\pi_1$ and $\pi_2$ being the {\em drop one part} functions:

\Comdiag{
        \& X\times Y \ar[dl,"\pi_1"'] \ar[dr,"\pi_2"] \\
    X \& \& Y \\
}

This is called a span and can also be turned into a little inline diagram as
$X \leftarrow (X\times Y) \rightarrow Y$.
Of course, there are other things we could put at the top of the fork, maybe some other
set $A$ that maps both ways: $X \leftarrow A \rightarrow Y$.
But $X\times Y$ is unique in this capacity because every $A$ and its span can be expressed
via a morphism to $X\times Y$. Here is the commutative diagram:

 \Comdiag{
        \& X\times Y \ar[dl,"\pi_1"'] \ar[dr,"\pi_2"] \\
    X \& \& Y \\
    \& A \ar[ul, "\forall f"] \ar[ur, "\forall g"'] \ar[uu, dashrightarrow, "f\times g"]
    }

For any $f$ and $g$, you could go the short way to map $A$ to $X$ or $Y$, but you are
guaranteed that there is also a pair of mappings $A\to (X\times Y) \to X$ and $A\to
(X\times Y) \to Y$ that are
the long commute to the same point.  Put differently, you can have an arbitrary
span composed of any object and mappings iff you can write down a map $A\to (X\times
Y)$. That any span can be expressed by going through $X\times Y$ makes that object a
sort of universal spanner.

One can generate other universal properties for other common diagrams and operations.
As per the names, the intent is that the universal properties do say something fundamental
about the relationship diagram, and that natural transformations do describe the
natural way to move from one domain to another.

\nocite{awodey:category}
\nocite{riehl:category}
\bibliographystyle{plainnat}
\bibliography{cat_notes}

\end{document}
