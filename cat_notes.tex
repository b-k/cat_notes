\documentclass[11pt]{article}
\usepackage{amssymb,amsmath,tikz-cd,xspace}
\usepackage[T1]{fontenc}

\begin{document}
\input{tex.macros}

\section{Bundles}
We're used to thinking in terms of sets comprising atomic elements, but
most of the things to follow are a bundle of parts:

\items{
    \item {\em Magma}: A set $S$ and a function $f:S\to S$.
    \item {\em Preorder}: A set $S$ and a binary relation $\leq$. (See flash cards for
the formal definition)
    \item {\em Graph}: A collection $(V, A, src, tgt)$:
    \items{
        \item $V$ a set (vertices)
        \item $A$ a set (arrows)
        \item $src:A\to V$ a function giving arrow heads
        \item $tgt:A\to V$ a function giving arrow tails
    }
}

Authors who write in object-oriented programming languages will recognize this
use of {\em object} to mean a structure of elements, some of which could be verbs.

\subsection{Those object-function pairs}

\items{
    \item {\em Magma}: A set $S$ and a function $f:S\to S$ (so $f(s)\in S$).
    \item {\em Semigroup}: A magma where $f$ is associative: $f(a, f(b,c))=f(f(a, b),c)$.
    \item {\em Monoid}: A semigroup with an identity element, $f(id)=id$.
    \item {\em Group}: A monoid where every element has an inverse: $\forall x\in S$, $\exists x_{inv}$ $\ni f(x, x_{inv}) = id$.
}

Associativity and identity are going to be part of the definition of categories,
so there won't be a lot of discussion of things on the list before monoids.
Groups give a strong sense of symmetry, and so are reserved for contexts where
reversibility or symmetry makes sense.

\section{Leveling up}

A function $f:A\to B$, aka a mapping, assigns a value to all of the elements in $A$ to
some element(s) in $B$. Note that if $A=\emptyset$ then you still have a function, though
it is vacuous.

We can draw this function via arrows, maybe

\Comdiag{
    a_1 \ar[rd] \& b_1\\
    a_2 \ar[r] \& b_2\\
    a_3 \ar[r] \& b_3\\
}

Here, the nodes in the graph are single elements in the sets.

The function notation $f:A\to B$ is itself a tiny graph. If there were also a function
$g:C\to B$, we could construct a diagram that looks like the above element-by-element graph:

\Comdiag{A \ar[r,"f"]\& B\\
C \ar[ru, "g"] 
}
So we've gone up a level: each node-arrow-node set could be cracked open to reveal the
sort of element-by-element diagram we had above. A closed function $f:A\to A$ maps to its
own set:

\Comdiag{A\ar[loop right]}

We've zoomed out from nodes being elements to being sets of elements; the natural next
step is to generalize the arrows to being sets of arrows. Consider the collections of all
mappings of the form $f:A\to B$. In the first diagram with three elements in $A$ and
three in $B$, the collection could have nine elements---or fewer, our choice. Sets
use capitals and their elements lower case $(s\in S)$, so let the collection of
lower-case functions be $F$; we now have

\Comdiag{A \ar[r,"F"]\& B}

I think of $F$ as a bundle of quivering wires, each with its own power. If this were anime
they'd be tentacles. But there is indeed a lot of power there. With \Re representing the
real numbers as usual, the diagram 
\Comdiag{\Re\ar[loop right]}
now represents all real-valued functions.





\end{document}
