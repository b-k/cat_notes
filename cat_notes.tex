\documentclass[11pt]{article}
\usepackage{amssymb,amsmath,tikz-cd,xspace, natbib,verse}
\usepackage[T1]{fontenc}

\begin{document}
\input{tex.macros}

\title{Category theory notes}
\maketitle

Right now the document only touches on and motivates what seem to be the key ideas
pushed by the basic category theory textbooks I've been reading. There is still a long
to-do list I am trying to get through: duality, more about natural transformations and how
they chain together, adjoints, probably more examples.

\section{Intro}
The question I've gotten from everybody I talk to about this:
why study category theory at all? Remember I'm a novice and this is
forward looking, but here are my impressions halfway through a few textbooks:

\items{
\item  I'm hoping it'll be a shortcut to learning more math. If I get this, then
algebraic topology should be metaphorical cake.
\item  By {\em category}, we mean {\em context}, and we're looking for patterns across
contexts: a lot of mathematical pursuits do the same thing over and over. Can
we formalize exactly how whatever two things are
actually the same thing? There is a promise that even theoretical work in real-world
contexts (i.e., my actual job) will benefit from better formalization of how
contexts relate.
\item Also, there is a trend in programming languages toward languages that use mappings
from a certain category, the monad, back to itself (aka an endofunctor).
\item From what I've read, a lot of philosophers work on it---there's some presumption that
it's not just about symbol shunting, but about finding `natural' and `universal'
properties of those patterns across disparate things.
\item  The textbook intros seem to motivate it via verbs and motion, as if you are touring from category
to category, but maybe it's the opposite. 
Maybe math is not `plug 4 into the square functions and we do some things and return 16'
but `the square of 4 is 16 and always has been and always will be.' The relationship
between these categories/contexts has always been there and we only need a way to see them.
Here's Robert Penn Warren on the matter:
}

\begin{verse}
Here is the shadow of truth, for only the shadow is true.\\
And the line where the incoming swell from the sunset Pacific\\
First leans and staggers to break will tell all you need to know\\
About submarine geography, and your father's death rattle\\
Provides all biographical data required for the {\em Who's Who} of the dead.

\dots In the distance, in {\em plaza, piazza, place, platz,} and square,\\
Boot heels, like history being born, on cobbles bang.

Everything seems an echo of something else.

\dots I watched the sheep huddling.  Their eyes\\
Stared into nothingness.  In that mist-diffused light their eyes\\
Were stupid and round like the eyes of fat fish in muddy water,\\
Or of a scholar who has lost faith in his calling. \dots

You would think that nothing would ever again happen.

That may be a way to love God.
\end{verse}

OK, now that I've answered that question, let's start from simple element-to-element
mappings and work up.

\section{Leveling up}\label{levelsec}

A function $f:A\to B$, aka a mapping, assigns a value to all of the elements in the set $A$ to
some element(s) in the set $B$. Note that if $A=\{\emptyset\}$ then you still have a function, though
it is vacuous.

We can draw this function via arrows, maybe

\begin{center}
\begin{tikzcd}[row sep=tiny]
    a_1 \ar[rd] & b_1\\
    a_2 \ar[r] & b_2\\
    a_3 \ar[r] & b_3
\end{tikzcd}
\end{center}

Here, the nodes in the graph are single elements in the sets, and the arrows describe
where each element goes.

The function notation $f:A\to B$ is itself a tiny graph. If there were also a function
$g:C\to B$, we could construct a diagram that looks like the above element-by-element graph, but with sets at the nodes instead of single elements:

\Comdiag{A \ar[r,"f"]\& B\\
C \ar[ru, "g"] 
}
So we've gone up a level: each arrow at this level could be cracked open to reveal
a bundle of arrows, its own element-by-element diagram like the one we had above.

A closed function $f:A\to A$ maps to its own set, so we'd draw it as a loop:

\Comdiag{A\ar[loop right]}

But we can bundle further. In the last diagram we had a single function. But
consider the collections of all
mappings of the form $f:A\to B$. The first diagram with three elements in $A$ and
three in $B$ could be rewired nine different ways. The full set of these mappings (aka
homomorphisms) is $\Hom{set}(A, B)$. But our collection doesn't have to be the whole set;
it's our choice. Sets use capitals and their elements lower case $(s\in S)$, so by
analogy, given lower-case functions $f$, let a collection of those be $F$; we would draw these
bundles as

\Comdiag{A \ar[r,"F"]\& B}

I think of $F$ as a bundle of quivering wires, each with its own power. If this were anime
they'd be tentacles. But there is indeed a lot of power there: with \Re\ representing the
real numbers, we can now summarize all real one-variable functions with the diagram 
\Comdiag{\Re\ar[loop right]}

This is the level at which category theory speaks. But we can go one step further, and put
this into a box as well, and now it is a single node in our graph. Let the above loop of
all functions $f:\Re\to\Re$ be $\bf R$, and let's keep its company with the set
the set of all functions on the natural numbers, $g:\Nat\to\Nat$, notated {\bf N}.

\Comdiag{{\bf N}\ar[r] \& {\bf R}}

Here, the arrow is a {\em functor}, which is a bundle of two sets of arrows at once:
one to map elements of $\Nat$ to elements of $\Re$, and one to map the functions
$\Nat\to \Nat$ to functions $\Re\to \Re$.

All these diagrams look alike, so for any diagram you come across in a textbook, it's
worth checking where we are in the pyramid:
\items{
    \item Node is a single element; arrow represents a mapping from one element to another
    \item Node is a set; arrow represents a set of mappings from one set to another, aka a function
    \item Node is a set; arrow represents a set of functions from one set to another
    \item Node is a set of objects with their own sets of functions; arrow represents a functor from one context to another
}

\section{Bundles}\label{bundlesec}
We're used to thinking in terms of sets comprising atomic elements, but mathematical
objects are typically a bundle of parts. Non-mathematicians think about the non-negative real numbers
$\Re^+$, but we have to start thinking about $\Re^+$ paired with addition and $\Re^+$
paired with multiplication as different things. Some bundles:

\items{
    \item {\em Magma}: A set $S$ and a function $f:S\to S$.
    \item {\em Preorder}: A set $S$ and a binary relation $\leq$. (See flash cards for
the formal definition)
    \item {\em Graph}: A collection $(V, A, src, tgt)$:
    \items{
        \item $V$ a set (vertices)
        \item $A$ a set (arrows)
        \item $src:A\to V$ a function giving arrow sources
        \item $tgt:A\to V$ a function giving arrow targets
    }
}

Authors who write in object-oriented programming languages will recognize this
use of {\em object} to mean a structure of elements, some of which could be verbs.

\subsection{Constraints}
These objects often have constraints attached, and what we name the pair $(S, f)$ depends on
how restrictive we want to be about the function:

\items{
    \item {\em Magma}: A set $S$ and a function $f:S\to S$.
    \item {\em Semigroup}: A magma where $f$ is associative: $f(a, f(b,c))=f(f(a, b),c)$.
    \item {\em Monoid}: A semigroup with an identity element for which $f(id)=id$.
    \item {\em Group}: A monoid where every element has an inverse: $\forall x\in S$,
            $\exists x_{inv}$ such that $f(x, x_{inv}) = id$.
}

As an example, if we allow zero in $\Re^+$, then $(\Re^+, +)$ is a monoid (no negative numbers, so
no inverses) and $(\Re^+, \cdot)$ is a group.

Associativity and identity are going to be part of the definition of categories,
so there won't be a lot of discussion of things on the list before monoids.
Groups are reserved for contexts where the sense of reversibility or symmetry make sense.

\subsection{Category characteristics}
See the flash cards for the formalization, but less formally,
a category is a bundle of elements as per the end of the chain of arrow abstractions. Bundle
together some things that will be nodes (herein, objects),
and the sets of mappings between some or all of them.

You get to pick which mappings you will put in the set of mappings, with a few simple conditions.
You don't have to include every mapping (herien, morphism) from node to node, which as above may cover all of one-variable high school algebra with a single arrow. But for every set we do have to include
the identity morphism $id_A:A\to A$ mapping every element of an object to itself.

It also has a composition function, so given $F:A\to B$
and $G:B\to C$, we have to include $G\circ F:A\to C$. I haven't yet seen this composition
function be anything but the obvious $g(f(a)) = c$, so this is not difficult, but in case
you get creative, the textbooks also require that $id_b\circ f(a)=f(a)$ and $g\circ id(b) = g(b)$.

Having these additional rules means we can't just throw out any old set of nodes and edges
and call it a category, but it's not hard to add these extra mappings as needed,
to the point that people usually don't even bother drawing the loops to mark the
identities.

Functors, as above, are a bundle of mappings of objects to objects plus mappings from
functions to functions.  We want them to represent a sensible transformation from one
context to another, so a functor $F$ has to have these constraints on how it maps
functions and objects:
\items{
    \item $F(f:A\to B) = F(f): F(A)\to F(B)$
    \item $F(id_A) = id_{F(A)}$
    \item $f\circ g = F(f)\circ F(g)$
}

\subsection{Starting to map}
The nodes in all the graphs above could be anything, including the bundles of elements from
Section \ref{bundlesec}.

Let ${\bf PrO}$ be the set of all preorders and ${\bf Graph}$ be the set of graphs. A
single preorder is a set of relations of the form $a \leq b$, and we could replace the $\leq$
symbol with an arrow, like $a\leftarrow b$, and we have another little inline graph. 
Doing the same for all of the elements in the preorder would generate a list of
$($nodes, arrows, sources, targets$)$ in ${\bf Graph}$.
That is, $f:{\bf PrO}\to {\bf Graph}$ makes as much sense as $f:\Re \to \Re$. Set up such
a mapping for every element of ${\bf PrO}$, and we have a category diagram with two
objects and one arrow representing all homomorphisms mapping all partial orders to graphs.

Categories are themselves a bundle like the ones from Section \ref{bundlesec}: a set
of objects paired with a set of morphisms between them. So they too could be a node in
one of the above diagrams, giving us the category of categories, ${\bf Cat}$.\footnote{Authors refer
to {\em the} category of categories, and I'm unclear whether an alternative cat-of-cats
exists or what it would look like, so we may be at the top here---there's no category of
categories of categories.}

Things are already getting self-referential: ${\bf PrO}\to {\bf Graph}$ is a
node-and-arrow graph representing an operation involving the set of node-and-arrow
graphs. For ${\bf Cat}$, we can show that ${\bf Cat}\to {\bf Graph}$, thus drawing a
graph for the category \{${\bf Cat}, {\bf Graph}$\} to represent how categories and
graphs relate.  Mathematicians love this stuff.

\section{Where commuting gets us} It's already potentially interesting that we can
express mappings between different concepts like orderings and graphs, but things don't get
interesting until we have multiple paths to the same place.


For example,
\citet{spivak:category} describes multisets as sets of elements $E$ with a set $B$ of
symbol names, and a map $\pi:E\to B$. So, for example, we could replace all the words in a
book with numbers, producing a sequence $e_1, e_2, \dots$, then have a dictionary
of the words, and $\pi$ tells us that $e_{250}$ maps to {\em dog}. We
may have another set of elements and symbols in another context, maybe a corpus in
another language, so we'll need an element mapping $f_1:E\to E'$ and a symbol mapping
$f_2:B\to B'$. See the flash cards for formalization.

Here is a diagram for mapping a multiset $(E, B, \pi)$ to another:
\Comdiag{E \ar[r, "f_1"] \ar[d, "\pi"'] \& E'  \ar[d, "\pi'"]\\
        B \ar[r, "f_2"'] \& B'
    }

The dictionary definition of {\em
commute} is {\em to exchange}; in math circles, {\em commutativity} typically refers to
reversing the order of things, like $x+y = y+x$. In this case, we can 
get from the original element set to the primed symbol set in two ways:
either apply the symbol map $\pi$ first to get to $B$ and then step from the unprimed to
primed corpus (calculating $f_2\circ\pi(e)$), or first step to the primed corpus via $f_1$ and then
map from $E$ to $B$ on that side (calculating $\pi\circ f_1(e)$).

When are both routes identical? This is often an interesting question in its own right.
For books and word lists in Portuguese and English, I'd guess there's a natural
way to map simple nouns: Finding that an element in the $P$ corpus is {\em cão},
then translating that word to the Portuguese-to-English dictionary to find that it means {\em dog}
is likely equivalent to the commuted procedure of first finding the element in the
English corpus matching our element of interest, then looking up that that element in
the $E$ corpus maps to {\em dog}. This is going to be easy and natural, while doing
the same with {\em saudade} is going to take some forcing.

A slight digression to further clarify commutative diagrams:
one way to nail down the relationship is to just cull down what is in your sets. The set
of books, $B$, can be cataloged either via the Library of Congress classification system
(a set of call numbers $L$), or the Dewey decimal system (another set of call numbers
$D$). Let us pair together call numbers referring to the same book, in the cross-product
space $D\times L$. The full cross product makes no sense, and
not all books have a classification in every system, so
out of the full cross of all pairs $(d, l)$,
we will write down the relatively small list of pairs that are about the same book, herein $D\times_B L$.
Finally, we have commutativity:
\Comdiag{D\times_B L \ar[r] \ar[d] \& L  \ar[d]\\
        D \ar[r] \& B
    }
There is information in the pairing, and there is even information in the list of books
and call numbers that can't be found from $D\times_B L$ and are now lost in the stacks.

These examples were at the set level, but one could do the same at the category level.
I'd used the word {\em natural} to describe the procedure to map across languages,
then find in dictionary; or identically find in dictionary then map across languages.
If there is a mapping from one context to another where (apply internal function,
then jump) produces the same outcome as (jump, then apply internal function), we call
the jump across categories a {\em natural transformation}.

Further, all these commutative diagrams chain together:

\Comdiag{E \ar[r, "f_1"] \ar[d, "\pi"'] \& E'   \ar[r, "f_3"] \ar[d, "\pi'"] \& E'' \ar[d, "\pi''"]\\
        B \ar[r, "f_2"'] \& B' \ar[r, "f_4"'] \&B''
    }

It is self-evident that if the left square and the right square commute, than any flow
from upper left to lower right will be equivalent. This becomes especially interesting
when we have mappings across different contexts: if we can map from multisets to preorders
and preorders to graphs, we just got a multiset-to-graph generator for free.

\subsection{Universal properties}
Consider the cross product of two sets, $A\times B$. There is an obvious mapping from
$A\times B \to A$ and $A\times B \to B$---just drop the unneeded $B$ or $A$ part. Here
is the shape of this as a diagram, with $\pi_1$ and $\pi_2$ being the {\em drop one
part} functions:

\Comdiag{
        \& X\times Y \ar[dl,"\pi_1"'] \ar[dr,"\pi_2"] \\
    X \& \& Y \\
}

This is called a span and can also be turned into a little inline diagram as
$X \leftarrow (X\times Y) \rightarrow Y$.
Of course, there are other things we could put at the top of the fork, maybe some other
set $A$ that maps both ways: $X \leftarrow A \rightarrow Y$.
But $X\times Y$ is unique in this capacity because every $A$ and its span can be expressed
via a morphism to $X\times Y$. Here is the commutative diagram:

 \Comdiag{
        \& X\times Y \ar[dl,"\pi_1"'] \ar[dr,"\pi_2"] \\
    X \& \& Y \\
    \& A \ar[ul, "\forall f"] \ar[ur, "\forall g"'] \ar[uu, dashrightarrow, "f\times g"]
    }

As noted in the diagram, this is trivial to generate as the ordered pair $(f(a), g(a))$.
For any $f$ and $g$, you could go the short way to map $A$ to $X$ or $Y$, but you are
guaranteed that there is also a pair of mappings $A\to (X\times Y) \to X$ and $A\to
(X\times Y) \to Y$ that are
the long commute to the same point.  Put differently, you can have an arbitrary
span composed of any object and mappings iff you can write down a map $A\to (X\times
Y)$. That any span can be expressed by going through $X\times Y$ makes that object a
sort of universal spanner.

One can generate other universal properties for other common diagrams and operations.
As per the names, the intent is that the universal properties do say something fundamental
about the relationship diagram, and that natural transformations do describe the
natural way to move from one domain to another.

\nocite{awodey:category}
\nocite{riehl:category}
\bibliographystyle{plainnat}
\bibliography{cat_notes}

\end{document}
